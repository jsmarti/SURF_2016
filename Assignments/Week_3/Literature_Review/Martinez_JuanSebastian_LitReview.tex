\documentclass{journal}
\usepackage[margin=0.6in]{geometry}
\usepackage[utf8]{inputenc}
\usepackage{indentfirst}

\hyphenpenalty=10000
\hbadness=10000

\title{\textbf{Be The Expert on Your Research Topic}\\Part 2: Writing a Literature Review}
\author{Juan Sebastian Martinez Carvajal}
\date{}

\begin{document}
\maketitle
\hrulefill

\section{\underline{Literature Review}}

A great variety of problems in science and engineering can be expressed as optimization problems, where a function that describes the operation of a system should be minimized or maximized, restricted to some constraints imposed by the system itself. For example, a supermarket may want to minimize its inventory purchases while keeping enough stock for its operation, or in a metal-forming shop, a manufacturer may want to maximize the performance of the manufacturing system adjusting certain parameters, like the forming temperature and the die geometry \cite{Huang2006}. Moreover, real optimization problems do not have trival or ``easy" solutions, since in many cases the optimization could target: multiple objectives; black box systems where the physics and mathematics of the process are not well know; expensive simulations or evaluations of objective functions \cite{Jones1998}; and the introduction of uncertainty in the output due to noise in the input data. In this context, a need for the design of a method for solving stochastic multi-objective optimization problems arises and becomes important in the science and engineering community.\\

One of the first issues in optimization that arises in real science and engineering problems is the lack of knowledge of the physics that rules the underlying process of the system being studied. This case is treated as a black-box system where output observations are available to build a model that could describe the system with the highest accuracy possible. Optimization under these circumstances has been approached from different perspectives, for example, advances in efficient global optimization of expensive simultations have been made through the analisis of various criteria of the problem, like the expected improvement (EI) of running the simulation in every run; also, stopping criteria for the sequencial optimization processes that rules black-box problems have also been developed \cite{Jones1998}. In this matter, the structure for solving black-box optimization problems is based on building response surfaces over observed data. The taxonomy of this process includes the use of different criteria to evaluate the sequencial process of optimization, in which the information acquisition scheme has been proven to be more efficient through the EI over other criteria, such as the probability of improvement and lower and upper confidence bounds \cite{Jones2001}.\\

This studies show how black-box systems have been approached in order to perform a sequencial process of optimization. However, 

\section{\underline{Mentor Check}}

\bibliographystyle{ieeetr}
\bibliography{SURF_2016}

\end{document}