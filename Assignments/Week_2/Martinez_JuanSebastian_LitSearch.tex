\documentclass{journal}
\usepackage[margin=0.6in]{geometry}
\usepackage[utf8]{inputenc}
\usepackage{indentfirst}

\hyphenpenalty=10000
\hbadness=10000

\title{\textbf{Be The Expert on Your Research Topic}\\Part 1: Conducting a Literature Search}
\author{Juan Sebastian Martinez Carvajal}
\date{}

\begin{document}
\maketitle
\hrulefill

\section{Task 1}
Regarding the summer research project \textit{Design Optimization of a Stochastic Multi-Objective Problem}, 10 related references were found; the following numbered citations correspond to the full citation for all the references:\\

\bibliographystyle{ieeetr}
\bibliography{SURF_2016}

\subsection{Questions}

\subsubsection{Reference 1}
\begin{enumerate}
	\item Reference \cite{Huang2009} is an article and its full citation is: V. L. Huang, S. Z. Zhao, R. Mallipeddi, and P. N. Suganthan, “Multi-Objective Optimization Using Self-Adaptive Differential Evolution Algorithm,” 2009 IEEE Congress on Evolutionary Computation (CEC’2009), pp. 190–194, 2009.
	\item To find this reference, the Purdue Libraries Tool was used and it was found on the IEEE Xplore database.
	\item In this article, an objective-wise multi-objective self-adapting differential evolution method is developed to solve multi-objective optimization problems. This is relevant to the project in the sense that it describes a method to solve multi-objective optimization problems, which is the main goal of the research project.
	\item The main finding of this article is the development of a new method for solving different multi-objetive optimization problems, where a trial vector is generated avoiding a time-consuming trial and error scheme, which was the common method used.
	\item In this article, one possible gap identified is how the mutation strategy in the genetic algorithm may impact the computational performance of the optimization process, due to high dimensionality in the input, which is a possibility considered in the research project. Another gap identified is the fact that the study presented does not consider uncertainty propagation.
\end{enumerate}

\subsubsection{Reference 2}
\begin{enumerate}
	\item Reference  \cite{Seeger2004} is a book and its full citation is: M. Seeger, Gaussian processes for machine learning., vol. 14. 2004.
	\item This book was a mentor recommendation.
	\item For the purpose of the project, the problem of optimization under uncertainty is addressed by a Bayesian interpretation of Gaussian processes and probability. This book contains the theory about Gaussian process regressions applied to machine learning techniques and provides some of the mathematic background needed for the project.
	\item The book is used as an important theoretical and mathematical reference for the concepts implemented in the project, because of this, a disussion about findings is not applicable.
	\item The book is used as an important theoretical and mathematical reference for the concepts implemented in the project, because of this, a discussion about limitations is not applicable.
\end{enumerate}

\subsubsection{Reference 3}
\begin{enumerate}
	\item Reference \cite{Davidson-Pilon2014} is a book and its full citation is: C. Davidson-Pilon, “Probabilistic programming and bayesian methods for hackers,” 2014.
	\item This book was a mentor recommendation.
	\item For the purpose of the project, an interactive GUI tool will be developed so that different experimentalists can use the probabilistic Bayesian method developed for solving multi-objective optimization problems. In this context, this book contains all the theory needed for the scientific programming tasks that are necessary for the project. 
	\item The book is used as an important theoretical and computational reference for the concepts implemented in the project, because of this, a disussion about findings is not applicable.
	\item The book is used as an important theoretical and computational reference for the concepts implemented in the project, because of this, a disussion about limitations is not applicable.
\end{enumerate}

\subsubsection{Referene 4}
\begin{enumerate}
	\item Reference \cite{Wang2016} is an article and its full citation is: H. Wang, G. Lin, and J. Li, “Gaussian process surrogates for failure detection: A Bayesian experimental design approach,” Journal of Computational Physics, vol. 313, no. September 2015, pp. 247–259, 2016.
	\item To find this reference, the Purdue Libraries was used and it was found on the Scopus database.
	\item This reference is relevant to the project as it approaches the problem of quantifying uncertainty through Gaussian surrogates and treats the problem of expensive computer models, where sampling the input state is a great concern. The research project involves the use of Gaussian process regressions in the process of designing optimization solutions under uncertainty, which is related to the concepts developed in the reference.  
	\item The main finding of this article is the construction of a failure-detection method based on Gaussian processes that estimates uncertainty.
	\item Regarding the article's results, the research project to be studied does not fill any specific "gap" or adress any limitation that the reference has in a direct way. Instead, the article probes that Gaussian processes are very useful to treat uncertainty in an input space, which is one for the project basis. Because of this, the reference is not listed as an article that supports the "gap"-filling objective of the project, but supports its methods for solving the proposed problem.
\end{enumerate}

\subsubsection{Reference 5}
\begin{enumerate}
	\item Reference \cite{Schaul2011} is an article and its full citation is: T. Schaul, Y. Sun, D. Wierstra, F. Gomez, and J. Schmidhuber, “Curiosity-Driven Optimization,” pp. 1343–1349, 2011.
	\item This reference was taken from the Purdue Libraries tool, using the IEEE Xplore database.
	\item This reference is relevant to the project as it addresses the problem of black box optimization where data point evaluations are expensive. This is a major concern in the research project that tries to fin optimal evaluation points of experiments that are very expensive and that do not have a specific mathematical of physical description.
	\item The main findings of this reference rely on the introduction of a novel technique to solve optimization problems through the principle of artificial curiosity. This was made again through the use of Gaussian processes.
	\item In this article, the problem of design optimization for black box models is addressed with an efficient computational method (curiosity-driven optimization) that can also treat uncertainty at the input through a Gaussian process. Their results show that computational cost were still high after all the process, which is a clear "gap" that the research project will try to overcome.
\end{enumerate}

\subsubsection{Reference 6} 
\begin{enumerate}
	\item Reference \cite{Li2014} is an article and its full citation is: B. Li, H. Cheng, H. Chen, and T. Jin, “Modeling Complex Robotic Assembly Process Using Gaussian Process Regression,” pp. 456–461, 2014.
	\item To find this reference, the Purdue Libraries Tool was used and it was found on the IEEE Xplore database.
	\item This article is relevant to the project as it is an instance of the research project to be implemented. In this reference, an optimization process is modeled through Gaussian process regressions where evaluation of inputs and experiments are costly; in addition, an stochastic environment for the system is considered.
	\item The main finding of this article was an algorithm to determine the optimal covariance function, that produces the best Gaussian process regression to solve the assembly problem proposed.  
	\item In this article, the main focus of the solution was the correct determination of a Gaussian process regression, and the best covariance function that could result in an optimal operation of the system being modeled. One "gap" identified related to this was the lack of consideration of uncertainty propagation in the input variables, which is considered in the research project to be studied.
\end{enumerate}

\subsubsection{Reference 7}
\begin{enumerate}
	\item Reference \cite{Pandita2016} is an article and its full citation is: P. Pandita, I. Bilionis, and J. Panchal, “Extending Expected Improvement for High-dimensional Stochastic Optimization of Expensive Black-Box Functions,” arXiv:1604.01147, p. 10, 2016.
	\item This article was a mentor recommendation.
	\item This article is relevant to the project in the sense that it is the basis for its development. It addresses the fact of considering uncertainty in high-dimensional optimization problems applying Bayesian optimization to stochastic optimization problems.
	\item The main findings of the article are the possibility of 
\end{enumerate}

\subsubsection{Reference 8}
\begin{enumerate}
	\item Reference \cite{Costa2006} is an article and its full citation is: E. O. Costa and A. Pozo, “A New Approach to Genetic Programming based on Evolution Strategies,” vol. 00, no. C, pp. 4832–4837, 2006.
	\item This article was found through the Purdue Libraries tool searching in the IEEE Xplore database.
	\item The article is relevant to the project as it develops new genetic algorithms trough the appliance of evolutionary strategies. In multi-objective optimization problems, genetic algorithms are used to obtain the Pareto front, which contains the optimal solutions to the problem. This technique could be useful as it is an alternative for solving this kind of problems.
	\item The main finding of this article is the development of a new approach to genetic programming, applying evolutionary techniques to the classic genetic programming algorithm.
	\item The methods used in this article have the limitation that uncertainty of the input is not considered. Moreover, high dimensionality could cause significant computational costs that are not considered. As stated earlier, this aspects are addressed by the research project and represent the "gap" to be filled.
\end{enumerate}

\subsubsection{Reference 9}
\begin{enumerate}
	\item Reference \cite{Guo2007} is an article and its full citation is: S.-M. Guo, “A Fast Multi-Objective Evolutionary Algorithm for Expensive Simulation Optimization Problems,” Second International Conference on Innovative Computing, Information and Control (ICICIC 2007), pp. 324–324, 2007.
	\item This article was found through the Purdue Libraries tool searching in the IEEE Xplore database.
	\item The article is relevant to the project as it develops a multi-objective evolutionary algorithm aimed to solve expensive simulation-based optimization problems. In this context, the research project targets the construction of the Pareto front with low computational costs; this article addresses this problem through an evolutionary perspective.
	\item The main finding in this reference is the construction of an evolutionary algorithm with high efficiency for multi-objective optimization problems that require the evaluation of the objective function through simulation.
	\item As stated before, the evolutionary algorithm is an alternative to the Bayesian approach taken by the research project, and does not contemplate uncertainty; this is a "gap" to be considered. Still, an evolutionary approach could also be useful in the project adding Gaussian processes and uncertainty propagation theory.
\end{enumerate}

\subsubsection{Reference 10}
\begin{enumerate}
	\item Reference \cite{Huang2006} is an article and its full citation is:  D. Huang, T. T. Allen, W. I. Notz, and N. Zeng, “Global optimization of stochastic black-box systems via sequential kriging meta-models,” Journal of Global Optimization, vol. 34, no. 3, pp. 441–466, 2006.
	\item This article was found through the Purdue Libraries tool searching in the IEEE Xplore database.
\end{enumerate}


\end{document}