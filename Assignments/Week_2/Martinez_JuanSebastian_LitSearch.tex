\documentclass{journal}
\usepackage[margin=0.6in]{geometry}
\usepackage[utf8]{inputenc}
\usepackage{indentfirst}

\hyphenpenalty=10000
\hbadness=10000

\title{\textbf{Be The Expert on Your Research Topic}\\Part 1: Conducting a Literature Search}
\author{Juan Sebastian Martinez Carvajal}
\date{}

\begin{document}
\maketitle
\hrulefill

\section{Task 1}
Regarding the summer research project \textit{Design Optimization of a Stochastic Multi-Objective Problem}, 10 related references were found; the following numbered citations correspond to the full citation for all the references:\\

\bibliographystyle{ieeetr}
\bibliography{SURF_2016}

\subsection{Questions}

\subsubsection{Reference 1}
\begin{enumerate}
	\item Reference \cite{Huang2009} is an article and its full citation is: V. L. Huang, S. Z. Zhao, R. Mallipeddi, and P. N. Suganthan, “Multi-Objective Optimization Using Self-Adaptive Differential Evolution Algorithm,” 2009 IEEE Congress on Evolutionary Computation (CEC’2009), pp. 190–194, 2009.
	\item To find this reference, the Purdue Libraries Tool was used and it was found on the IEEE Xplore database.
	\item In this article, an objective-wise multi-objective self-adapting differential evolution method is developed to solve multi-objective optimization problems. This is relevant to the project in the sense that it describes a method to solve multi-objective optimization problems, which is the main goal of the research project.
	\item The main finding of this article is the development of a new method for solving different multi-objetive optimization problems, where a trial vector is generated avoiding a time-consuming trial and error scheme, which was the common method used.
	\item In this article, one possible gap identified is how the mutation strategy in the genetic algorithm may impact the computational performance of the optimization process, due to high dimensionality in the input, which is a possibility considered in the research project. Another gap identified is the fact that the study presented does not consider uncertainty propagation.
\end{enumerate}

\subsubsection{Reference 2}
\begin{enumerate}
	\item Reference  \cite{Seeger2004} is a book and its full citation is: M. Seeger, Gaussian processes for machine learning., vol. 14. 2004.
	\item This book was a mentor recommendation.
	\item For the purpose of the project, the problem of optimization under uncertainty is addressed by a Bayesian interpretation of Gaussian processes and probability. This book contains the theory about Gaussian process regressions applied to machine learning techniques and provides some of the mathematic background needed for the project.
	\item The book is used as an important theoretical and mathematical reference for the concepts implemented in the project, because of this, a disussion about findings is not applicable.
	\item The book is used as an important theoretical and mathematical reference for the concepts implemented in the project, because of this, a discussion about limitations is not applicable.
\end{enumerate}

\subsubsection{Reference 3}
\begin{enumerate}
	\item Reference \cite{Davidson-Pilon2014} is a book and its full citation is: C. Davidson-Pilon, “Probabilistic programming and bayesian methods for hackers,” 2014.
	\item This book was a mentor recommendation.
	\item For the purpose of the project, an interactive GUI tool will be developed so that different experimentalists can use the probabilistic Bayesian method developed for solving multi-objective optimization problems. In this context, this book contains all the theory needed for the scientific programming tasks that are necessary for the project. 
	\item The book is used as an important theoretical and computational reference for the concepts implemented in the project, because of this, a disussion about findings is not applicable.
	\item The book is used as an important theoretical and computational reference for the concepts implemented in the project, because of this, a disussion about limitations is not applicable.
\end{enumerate}

\subsubsection{Referene 4}
\begin{enumerate}
	\item Reference \cite{Wang2016} is an article and its full citation is: H. Wang, G. Lin, and J. Li, “Gaussian process surrogates for failure detection: A Bayesian experimental design approach,” Journal of Computational Physics, vol. 313, no. September 2015, pp. 247–259, 2016.
	\item To find this reference, the Purdue Libraries was used and 
\end{enumerate}


\cite{Schaul2011}, \cite{Li2014}, \cite{Pandita2016}, \cite{Costa2006}, \cite{Guo2007}, \cite{Huang2006}
\end{document}