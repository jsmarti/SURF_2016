\documentclass[a4paper]{article}
\usepackage[margin=0.79in]{geometry}
\usepackage[utf8]{inputenc}
\usepackage{fontspec}
\usepackage[dvipsnames]{xcolor}
\setmainfont{Arial}

\hyphenpenalty=100000
\hbadness=100000
\setlength{\parindent}{0pt}

\begin{document}
{\footnotesize
  \textbf{The Summer Undergraduate Research Fellowship (SURF) Symposium\\
    4 August 2016\\
    Purdue University, West Lafayette, Indiana, USA\\
    }
}
\begin{center}
{\Large \textbf{Design Optimization of a Stochastic Multi-Objective Problem: Gaussian Process Regressions for Objective Surrogates\\}}
{\normalsize Piyush Pandita, Ilias Bilionis\\
School of Mechanical Engineering, Purdue University\\
Juan S. Martinez\\
Department of Electrical and Electronic Engineering, Universidad de los Andes\\
}
\end{center}
{\large \textbf{ABSTRACT}\\}

{\normalsize \textcolor{ForestGreen}{Global optimization of expensive, multi-modal and noisy multi-objective functions is a common problem that comes up frequently in various areas of computational and experimental research. Along with being expensive to evaluate and noisy, we are also particularly interested in cases where the objective function under consideration is a black box and does not provide gradient information of the quantity of interest (QoI) w.r.t to the inputs. }\textcolor{red}{Because of the high cost of every single evaluation of such objective functions, we can only obtain a limited number of evaluations of the objective function. This necessarily induces epistemic uncertainty (lack of knowledge due to limited data) on our problem. }\textcolor{Purple}{The Bayesian approach provides a natural framework to approach this problem, building Gaussian process  surrogates that model the objective functions and allow a sequential optimization process. The method applies the expected improvement (EI) information acquisition function in each iteration to perform an efficient global optimization (EGO) and discover the Pareto front of the problem, which contains the set of optimal solutions of the objectives. }\textcolor{TealBlue}{We implemented this method in a NanoHUB tool and tested it with synthetic examples and real observations of an expensive experiment, where it proved to be efficient in finding the corresponding set of optimal solutions for each problem. }This methodology demonstrates the efficiency of the Bayesian approach for handling epistemic uncertainty in multi-objective optimization problems, dealing with limited observations and avoiding a high amount of evaluations of time-consuming codes or expensive experiments.\\}

{\large \textbf{KEYWORDS}\\}

{\normalsize Gaussian process, Multi-Objective, Optimization, Regression, Surrogate, Uncertainty.\\}

{\large \textbf{MENTOR CHECK}\\}

{\normalsize Throughout the mentor meeting, only major changes were made to the motivation section and the problem statement section. The changes were made to make more emphasis on the purpose of the project, as a solution of common real life problems that arise in computational and experimental research. The remaining sections didn't have major changes of content. Finally, the mentor was pleased with the structure of the abstract and its contents.}

\end{document}
