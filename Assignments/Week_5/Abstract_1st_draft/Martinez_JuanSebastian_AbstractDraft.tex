\documentclass[a4paper]{article}
\usepackage[margin=0.79in]{geometry}
\usepackage[utf8]{inputenc}
\usepackage{fontspec}
\usepackage[dvipsnames]{xcolor}
\setmainfont{Arial}

\hyphenpenalty=10000
\hbadness=10000
\setlength{\parindent}{0pt}

\begin{document}
{\footnotesize
  \textbf{The Summer Undergraduate Research Fellowship (SURF) Symposium\\
    4 August 2016\\
    Purdue University, West Lafayette, Indiana, USA\\
    }
}
\begin{center}
{\Large \textbf{Design Optimization of a Stochastic Multi-Objective Problem: Gaussian Process Regressions for Objective Surrogates\\}}
{\normalsize Piyush Pandita, Ilias Bilionis\\
School of Mechanical Engineering, Purdue University\\
Juan S. Martinez\\
Department of Electrical and Electronic Engineering, Universidad de los Andes\\
}
\end{center}
{\large \textbf{ABSTRACT}\\}

{\normalsize \textcolor{ForestGreen}{Optimization under uncertainty of multi-objective systems is a common goal in many engineering and science problems, where the main objective is to find the values of input variables that maximize or minimize multiple outputs. }\textcolor{red}{These optimization processes are often tied to epistemic uncertainty (lack of knowledge due to limited data), as the systems to be analyzed correspond to expensive experiments or code simulations that can only be performed a limited number of times, moreover, the physics that describe the systems are not well known and a deterministic approach cannot be taken. }\textcolor{Purple}{This study develops a stochastic Bayesian approach to the problem, building Gaussian process regressions to build objective surrogates that model the objective functions and allow a sequencial optimization process. The method applies the expected improvement (EI) criteria in each iteration to perform an efficient global optimization (EGO) and discover the Pareto front of the problem, which contains the set of optimal solutions of the objectives. }\textcolor{TealBlue}{The method was implemented in a web tool, for it to be used by real users; it was tested with synthetic examples and real observations of an expensive experiment, where it proved to be efficient in finding the corresponding set of optimal solutions for each problem. }This methodology demostrates how a stochastic approach can handle uncertainty in multi-objective optimization problems, dealing with limited observations and avoiding a high amount of evaluations of time-consuming codes or expensive experiments.\\}

{\large \textbf{KEYWORDS}\\}

{\normalsize Gaussian process, Multi-Objective, Optimization, Regression, Surrogate, Uncertainty.\\}

{\large \textbf{MENTOR CHECK}\\}

\end{document}
